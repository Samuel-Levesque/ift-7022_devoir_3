Plusieurs améliorations ou changements pourraient être effectués pour réaliser ce projet, en voici quelques uns:

\begin{enumerate}
\item Lors de la création du corpus, on considère l'échange de 3 textos comme un gros bloc de texte. Il pourrait être intéressant de tenter de trouver une relation entre les échanges qui pourrait aider à faire une meilleure classification.

\item Un gros problème du projet réalisé est le problème de \emph{"runtime"} du code. En effet, l'ajout de variables de sentiment dans les attributs augmente terriblement le temps d'exécution du code. Il faudrait revoir en entier les fonctions qui y sont associées pour pouvoir tester plus de paramètres lors de notre recherche en grille.

\item Un travail plus exhaustif et robuste pourrait être fait sur les binettes ajoutées comme variables. On pourrait faire un modèle statistique qui analyse toutes les chaînes de caractères spéciaux et qui détermine le degré de signification de chacune pour toutes nos classes au lieu d'utiliser une méthode plus ad hoc et manuelle comme nous avons utilisée.

\item Ajout d'une étape de \emph{stemming/lemmatisation} pour voir si elle pourrait apporter des performances accrues malgré notre intuition initiale.
\end{enumerate}