Ce fut un projet très intéressant qui permet d'appliquer plusieurs connaissances en traitement de langue naturelle. Il permet d'appliquer les notions de classification de textes, d'analyse de sentiment et les expressions régulières.
Également, contrairement aux autres projets réalisés au cours de la session, ce projet laissait libre cours à notre imagination et nous avions la liberté d'expérimenter avec plusieurs approches pour résoudre un même problème.

L'utilisation d'emojis et de binettes pour classer les échanges de textos était également très intéressante puisqu'elle n'a pas été vue en classe, mais apporte tout de même une quantité non-négligeable d'information à notre modèle.

Le projet fut également très formateur pour les auteurs du code Python qui contrairement à leur habitude, ont tenté de faire un code plus lisible et réutilisable pour d'autres applications. 
