Pour parvenir à réaliser la tâche, on utilisera un objet de compte fourni dans la librairie \emph{sklearn} et d'autres variables décrites dans la section \nameref{sec:analyse_prelim}. 
Ensuite encore avec \emph{sklearn}, nous utiliserons un modèle de classification pour prédire l'émotion dégagée dans les échanges de textos.

Pour développer et tester notre modèle, le corpus d'entraînement fourni dans \emph{train.txt} a été séparé en corpus d'entraînement (80\%) et de test(20\%)  pour pouvoir évaluer la performance hors-échantillon de notre modèle. 

Ainsi, tous les paramètres seront optimisés par une validation croisée à K-plis sur notre corpus d'entraînement (80\% des données) et on valide la performance du modèle sur le corpus de test.

Après optimisation de nos hyper-paramètres, le modèle est calibré avec l'ensemble du corpus pour mise en production.